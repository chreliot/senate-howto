\DocumentMetadata{
pdfstandard={UA-2,A-4f},
tagging=on,
lang=en,
tagging-setup=
  {math/setup=mathml-SE,
  extra-modules=verbatim-alt}
}   % default set-up as of 2025-06-10
\documentclass[12pt]{article}

%%%%%%%%%%%%%
% VARIABLES %
%%%%%%%%%%%%%


%%%%%%%%%%%%%%%
% HEAD MATTER %
%%%%%%%%%%%%%%%

%%% SET DATES %%%

\usepackage[american]{datetime2}
% \DTMsavedate{meetingdate}{2025-08-14}
% \DTMsavedate{minutesdate}{}

%%% LOAD REQUIRED PACKAGES %%%

\usepackage{hyperref}

\usepackage{enumitem} 
\let\tightlist\relax % to deal with Pandoc's insertion


\usepackage{fontspec} \defaultfontfeatures{Ligatures={Common},
% Numbers=Lining
}
\setromanfont[Scale=1.0]{Adobe Garamond Pro}
\setsansfont[Scale=1.0]{Helvetica}
\setmonofont[Scale=1.0]{Courier}

\usepackage[ % requires LuaLaTeX
  driver=luatex,
  margin=1.4in
]{geometry}

\usepackage{fancyhdr} \pagestyle{fancy}
\fancyhead{} \fancyhead[LE,RO]{Senate How-To (\the\year)}

\usepackage{sectsty}
\sectionfont{\normalsize}

\usepackage{textcase}
% \usepackage{pdfpages} % for including PDF attachments

\usepackage{tocloft} % for customizing a compact table of contents 
% Customize apperance of ToC:
 \setlength{\cftbeforetoctitleskip}{-1em} % remove 1em of vertical space before ToC
\setlength{\cftbeforesecskip}{0mm} % compress vertical spacing between entries
\renewcommand{\cftsecfont}{\mdseries} % change section entries from bold to regular
\renewcommand\contentsname{} % cut out the heading "Contents" before ToC
\renewcommand{\cftsecleader}{\mdseries\cftdotfill{\cftdotsep}} % add dot fill?
\renewcommand{\cftsecpagefont}{\mdseries} % change page numbers from bold to regular
% \renewcommand{\cftsetpnumwidth}{0em} % less space to the left of the page numbers


%%%%%%%%%
% FIXES %
%%%%%%%%%


% % Avoid undefined \tightlist error from Pandoc
\providecommand{\tightlist}{}
\setitemize{noitemsep,topsep=0pt,parsep=0pt,partopsep=0pt}


\hypersetup{
pdftitle={Senate How-to},
pdfauthor={Christopher Eliot},
pdfkeywords={Hofstra, Senate, introduction},
pdflang={English},
linktocpage=true
}

%%%%%%%%%%%%
% HEADINGS %
%%%%%%%%%%%%

\begin{document}

\thispagestyle{empty}

%%%%%%%%%
% TITLE %
%%%%%%%%%


\begin{center}
  \textbf{\large Senate How-To} \\ Hofstra University Senate (\the\year) \\ (\emph{Draft of \today} )
\end{center}


%%%%%%%%%%%%%%%%
% INTRODUCTION %
%%%%%%%%%%%%%%%%

\noindent This is an informal introduction to Hofstra University's University
Senate, written especially to orient new senators. (This introduction
does not itself constitute official policy but will point out where
official policies can be found.)


%%%%%%%%%%%%%%%%%%%%%
% TABLE OF CONTENTS %
%%%%%%%%%%%%%%%%%%%%%

\begin{center}
\begin{minipage}[c]{4.25in} % Compress the ToC to width of half page-width
\tableofcontents
\end{minipage}
\end{center}


%%%%%%%%%%%%%%%
% BODY MATTER %
%%%%%%%%%%%%%%%

\section{What is all this for?}\label{what-is-all-this-for}

We joined the university, pretty much all of us, to work on things other
than academic policy. But we all want to do well here, and we want the
university to do well. Having a set of guidelines governing how we do
things can help everyone do what they want to do here well. Ideally,
those guidelines---those \emph{policies}---set us up to teach and learn
and engage with one another happily and productively, without onerous
encumbrances. Moreover, ideally the policies are ones we can support,
because we or our elected representatives had a hand in crafting them or
in keeping them around.

\section{Why is there a University
Senate?}\label{why-is-there-a-university-senate}

Accordingly, Hofstra's Faculty Statute VII establishes ``a University
Senate, convened by the Faculty.'' Our Senate has ``general powers of
supervision over all academic matters concerning the University as a
whole.'' And it can take up ``any other matters referred to it by the
Board of Trustees, the University President, or the Provost of the
University'' (FS VII).

\section{What is ``shared
governance''?}\label{what-is-shared-governance}

We sometimes refer to the ecosystem of activities, bodies, and
procedures around the University Senate as Hofstra's ``shared
governance'' apparatus. The term derives from principles developed in
the 1960s concerning the governance of higher education.

\href{https://www.aaup.org/reports-publications/aaup-policies-reports/topical-reports/statement-government-colleges-and}{A
statement of core principles} was written and adopted in 1966 by the
American Association of University Professors (the national
organization), the American Council on Education, and the Association of
Governing Boards of Universities and Colleges. The statement proposes
areas of responsibility for each of the interdependent parts of
universities, including their Board, Faculty, President, et al.,
highlighting the importance of communication among the parts. The
statement mentions the idea of ``a faculty-elected senate or council for
larger divisions or the institution as a whole''; it brings that up as
part of setting out ideals for faculty participation in institutional
governance, in communication and collaboration with other parties.

Hofstra's Senate is faculty-convened (FS VII A.1), and faculty members
structurally have a majority. But ours is not a ``faculty senate,'' in
that administrators, students, and staff also meaningfully participate
and vote.

One might think of Hofstra's shared governance apparatus as including at
least the Senate and its committees, the meetings of the full faculty,
the process of transmitting resolutions between the faculty and the
University President and Board of Trustees, and ongoing consultation
between the Faculty Leadership Group and the University President and
Provost.

\section{What are the Senate's areas of
responsibility?}\label{what-are-the-senates-areas-of-responsibility}

Besides supervising ``all academic matters concerning the University as
a whole'' and dealing with matters referred to it, the Senate is
empowered to ``review and provide recommendations on policies and
initiatives impacting academic matters.'' And, besides, it can
``generate initiatives regarding academic matters'' (FS VII 3). So, it
supervises, reviews, and initiates. And it does those things with both
academic matters and anything impacting academic matters. The result can
be policies, but also reports and other initiatives.

\section{Which policies does the Senate work
on?}\label{which-policies-does-the-senate-work-on}

The main institutional, academic and academic-related policies the
Senate works on are in three places:

\begin{itemize}
\tightlist
\item
  \textbf{\href{https://www.hofstra.edu/senate/faculty-policy-series.html}{The
  Faculty Policy Series}}: The FPS sets out policies and procedures for
  faculty members. There are currently about 62 of them, numbered
  between 1 and 99. More workaday than the Statutes---less fundamental
  and structural and more procedural---they range widely in topic, from
  instructional matters like handling academic integrity violations and
  grade appeals; to research matters like conflict of interest,
  copyright, and misconduct policies; to organizational matters like the
  responsibilities of department chairs and how adjunct faculty are
  hired; to employment matters like research leave and sick leave.
\item
  \textbf{\href{https://bulletin.hofstra.edu}{The Bulletins}}: Published
  annually, the Bulletins come in several forms: Undergraduate,
  Graduate, and Law, with a new edition for each session. They are fully
  online. Besides course and program descriptions, each Bulletin
  outlines academic policies in a student-oriented way. The Bulletin
  tells students what they can expect from the university and what will
  be required of them.
\item
  \textbf{\href{https://www.hofstra.edu/sites/default/files/2024-04/facultystatutes.pdf}{The
  Faculty Statutes}}: These nine items (plus two more moved or
  withdrawn), numbered Roman I to XI, represent general, foundational
  policies. They establish the university's administrative structure,
  determine who is on the Faculty and how they're appointed to it,
  specify how faculty meetings shall be conducted, and create the Senate
  itself.
\end{itemize}

Much of the work of the Senate involves supervising the FPS and the
general, university-wide portions of the Bulletin. The FPS is the main
place the Senate---with the approval of further levels---enshrines new
and updated policies.

Most of the Senate's business concerns policies for Hofstra's schools
\emph{other than} the Zucker School of Medicine and the Deane School of
Law. The medical school and the law school each have a representative on
the Senate, and their faculty serve in committee roles---especially on
Faculty Affairs and Planning \& Budget, where business can affect
them---but those schools enjoy autonomy for their academic policies.

Occasionally, the Senate also passes items that don't fit neatly into
the FPS or Bulletin, or even the Statutes, like a communication to the
University President or guidance to its Registrar.

\section{What about university policies other than
FPS/Bulletin/FS?}\label{what-about-university-policies-other-than-fpsbulletinfs}

Unlike at some other institutions, an area Hofstra's Senate is
\emph{not} involved in is determining faculty, staff, or
student-employee working conditions and compensation. Those are the
domain of Hofstra's collective bargaining units/unions and
administration. For faculty, Hofstra's chapter of the American
Association of University Professors negotiates agreements that are
codified in the current \href{https://aaup-hofstra.org}{Collective
Bargaining Agreement}, available on the AAUP chapter website.

In practice, this distinction between the appropriate domain of the
Senate and the appropriate domain of collective bargaining can be blurry
because there are Faculty Policy Series documents on topics like ``Work
Above Base Load'' and ``Outside Employment for Faculty.'' However, AAUP
and administration have negotiated that if there are any conflicts
between the FPS and the CBA, the CBA ``will control''
(\href{https://aaup-hofstra.org}{CBA} 3.2). That is, the CBA always has
precedence.

Note also that, separately, Hofstra University's Office of Community
Standards (within the larger, administrative Division of Student
Enrollment, Engagement, and Success) produces the
\href{https://sites.google.com/hofstra.edu/guide-to-pride/home}{\emph{Guide
to Pride}} for students. It focuses on non-academic community standards
but also pulls in academic policies from the Faculty Policy Series.

Hofstra also has a variety of other, non-academic policies like those
listed \href{https://www.hofstra.edu/about/policies.html}{here},
including safety and security plans, record-retention policy, and the
like. Many of these are beyond domain of the Senate, though in certain
cases where they impinge on academics, it can be appropriate for the
Senate to weigh in on them. For example, the Senate By-laws make the
Chemical Hygiene Plan the responsibility of the Special Committee on
Environmental Health and Safety.

\section{How is the Senate
structured?}\label{how-is-the-senate-structured}

The Senate has two main components: a deliberative assembly and a
collection of committees.

Senate meetings---the gatherings of the assembled Senate
itself---normally occur at least once a month during the academic year.
These meetings of the full Senate normally focus on considering business
passed to it by the Senate's committees via the Executive Committee.

Committees do the work of reviewing, analyzing, drafting, and
deliberating over policies and initiatives. As of this writing, the
Senate has 15 committees.

The Senate has three main kinds of permanent committees: standing
committees, special committees, and subcommittees. (The Senate can also
form temporary, ad hoc committees.)

\textbf{Standing committees} (other than the Executive Committee) cover
general, main areas of Senate concern. Faculty Statute VII (D.1)
establishes six standing committees:

\begin{itemize}
\tightlist
\item
  The Undergraduate Academic Affairs Committee (UAAC)
\item
  The Graduate Academic Affairs Committee (GAAC)
\item
  The Planning and Budget Committee (P\&B)
\item
  The Faculty Affairs Committee (FAC)
\item
  The Student Affairs Committee (SAC)
\item
  The Senate Executive Committee (SEC)
\end{itemize}

The standing committees meet regularly; they are normally scheduled to
meet at least once a month during the academic year, and they sometimes
hold additional meetings.

The Senate Executive Committee consists of the chairs of the Senate's
standing committees, the President of the Senate, and the Provost (or
their designate). It normally communicates with standing committees
through their chairs. Its main role is to facilitate the Senate.

\textbf{Special committees} are permanent committees appointed by the
Senate or by one of its standing committees. They are charged with
undertaking investigations or recommending policy or action in specific
areas of Senate concern. Special committees appointed by the Senate
report to the Senate through the Executive Committee (SEC). They may be
asked by the Executive Committee to report on specific items directly to
the Senate. As of this writing, the By-laws specify five special
committees:

\begin{itemize}
\tightlist
\item
  Communications, Outreach, and Election Committee (COEC)
\item
  Special Committee on Grievances
\item
  Special Committee on Athletic Policy
\item
  Special Committee on the Academic Calendar
\item
  Special Committee on Environmental Health and Safety
\end{itemize}

\textbf{Subcommittees} are effectively special committees constituted by
one of the standing committees; accordingly, they report to their parent
standing committees rather than directly to the Senate, usually once per
semester. There are currently four subcommittees listed in the By-laws:

\begin{itemize}
\tightlist
\item
  Academic Review Committee (reports to UAAC)
\item
  Committee on the Library (reports to P\&B)
\item
  Committee on Environmental Priorities (reports to P\&B)
\item
  Committee on Education and Research Technology (reports to P\&B)
\end{itemize}

\section{What is the Senate's main
workflow?}\label{what-is-the-senates-main-workflow}

The Senate's work is nearly all done by its committees and therefore by
the voting members of the committees.

The committees review policies and keep an eye out for issues. The
issues they take up can come from a lot of different sources: from
committee members, from the committee's chair, from a subcommittee, as a
referral from the Senate Executive Committee or the Senate itself, from
the committee's adviser or another administrator, or from anyone else on
campus. The sources are unrestricted. However, the committee's chair
sets the agenda for the committee, and the committee itself determines
what issues it will take up and not take up.

Committees then draft or revise policy proposals and other items they
want to put before the Senate. When the committee determines that an
item is ready to go forward, the committee votes on it.

An item passed by a committee goes to the Senate Executive Committee.
The committee's chair brings it to the SEC. The SEC then also votes on
whether to take the item forward to the Senate---perhaps with
modifications---or to return it to the committee, or something else.

Then, at Senate meetings, the Chair of the SEC (who also serves as the
Senate President) presents items the SEC has passed to the Senate. The
Senate may discuss proposals and then vote on approving them (see FS VII
A.5).

As the Senate is a deliberative assembly, members can debate, raise
questions about, and even suggest adjustments to proposals that come
from committees. However, if it is to work efficiently, the Senate
cannot also be a good venue for extended debate or policy-writing.
Moreover, it is too easy for sentences written by large groups on the
fly to have unintended consequences or oversights. So, if many questions
or objections are raised, or substantial suggestions for revisions are
made, it can be appropriate for the Senate to send an item back to
committee instead of approving it.

When items are passed by the Senate, they normally go to the full
faculty as action items (or sometimes information items).

\section{What happens to items passed by the
Senate?}\label{what-happens-to-items-passed-by-the-senate}

Action items passed by the Senate normally go to the full faculty for
its approval at one if its four annual working meetings (see FS IX).

Items passed by the full faculty are then sent, via intermediate review
by the Provost and the Office of General Counsel, to the University
President and Board of Trustees. Items passed by the Senate only become
policies when approved and signed by the University President.

The standard approval sequence for Senate proposals is therefore:
(subcommittee \rightarrow{} ) standing or special committee
\rightarrow{} Senate Executive Committee \rightarrow{} University Senate
\rightarrow{} Faculty \rightarrow{} General Counsel \rightarrow{}
Provost \rightarrow{} University President ( \rightarrow{} Board of
Trustees).

However, the flow of passed motions is not always linear; items are
sometimes returned to previous stages, including all the way back to the
originating committees.

The University President and the Speaker of the Faculty normally notify
the Senate and full faculty when they receive notice that the University
President has signed something passed by the Senate and Full Faculty.

\section{Must all Senate items come from its
committees?}\label{must-all-senate-items-come-from-its-committees}

Though it is not the standard workflow for fully reviewing initiatives,
sometimes items are introduced at Senate or full faculty meetings as
motions (subject to guidelines in FS VII, FS X, and \emph{Robert's
Rules}). In other words, it's not \emph{required} that items coming
before the Senate have their origin in a committee. But because the
Senate isn't equipped to evaluate the full implications or complexities
of many proposals, new business introduced at Senate may often be
referred to the appropriate committee for analysis, or at least to the
Executive Committee.

But also, the Senate's rules specify that ``before a vote may be taken
on an item presented for action, senators must have had at least two
working days published notice'' (FS VII C.2.a). So, action items must be
circulated in advance. The Senate's Senior Support Specialist usually
finalizes and circulates the Senate's agenda on the Thursday before
Monday Senate meetings. So, items that need ``published notice'' need to
go to her before then. Ideally, they would also go to the Senate
Executive Committee, which sets the meeting agenda; it normally meets
one week before Senate meetings.

\section{What rules govern the
Senate?}\label{what-rules-govern-the-senate}

The University Senate is established and primarily governed by
\href{https://www.hofstra.edu/sites/default/files/2024-04/facultystatutes.pdf}{Faculty
Statute VII, ``The University Senate''}. A current copy of all the
Statutes including FS VII is linked and available on the
\href{https://www.hofstra.edu/senate/}{Senate website}.

The Senate's
\href{https://www.hofstra.edu/sites/default/files/2022-04/senatebylaws.pdf}{By-laws}
(thus titled) mostly concern its individual committees, though Section I
(``General'') also lists a handful of broader rules.

Faculty Statute VII also stipulates that except in matters covered by
the Statute, the current edition of \emph{Robert's Rules of Order} shall
be considered binding. In practice, the Senate relaxes some requirements
like that speakers must stand up, but the meeting structure and
proceedings generally conform to the \emph{Rules}.

Of course, in the background, Hofstra's general policies, including
those in the
\href{https://www.hofstra.edu/senate/faculty-policy-series.html}{Faculty
Policy Series} related to community standards and professional conduct,
continue to apply, as do laws.

\section{Who is in the Senate?}\label{who-is-in-the-senate}

There are currently about 41 members of the Senate. There are two
categories of voting members of the Senate: elected and ex-officio
members.

Various constituencies get to elect senators to represent them on the
Senate: regular professors from each of the schools, adjunct faculty as
a unit, staff/Local 153, and students. Some of the faculty elections are
managed by the Senate through its Communications, Outreach, and Election
Committee. (See FS VII B.2.)

Other people have a role in the Senate because of their roles elsewhere.
These ex-officio members include the Provost, one academic dean, the
Chief Diversity and Inclusion Officer, a representative designated by
the Vice President of Student Enrollment, Engagement and Success, the
Speaker and Vice Speaker of the Faculty, the Chair of the Chairs'
Caucus, and the student who is President of the Student Government
Association. By virtue of serving in their roles, they are full voting
members of the Senate. (See FS VII B.1.)

\section{When does the Senate meet?}\label{when-does-the-senate-meet}

The Senate normally meets once a month during the academic year,
typically on Monday afternoons at 1 PM. The Senate may also have special
meetings, though that hasn't happened in recent memory. (See FS VII
C.2.) The current year's meeting schedule is available through the
\href{https://www.hofstra.edu/senate/}{Senate website}. Invitees will
receive Outlook calendar invitations well in advance, by email, from the
Senate's Senior Support Specialist.

\section{Who may attend Senate
meetings?}\label{who-may-attend-senate-meetings}

Elected, ex officio senators, and senators-at-large are invited to
attend. Members of the faculty, administrators, chairpersons, students,
and staff may also observe meetings of the Senate. These others may,
upon invitation of the Senate President, and with the consent of the
body, participate in its deliberations, but they may not vote. (See FS
VII C.2.d.) The Senate has a quorum only when at least half its
\emph{elected} members are present (FS VII C.2.c).

\section{Who coordinates the Senate?}\label{who-coordinates-the-senate}

Most of the logistical work required to operate the Senate is performed
by the Senate's Senior Support Specialist. sends the invitations,
organizes the agendas, puts out the name placards, coordinates the Zoom
during meetings, and lots else. As of this writing, the Senate's Senior
Support Specialist is Caroline Schreiner
(\href{https://www.hofstra.edu/senate/}{contact information}).

The President of the University Senate runs it. The Senate president is
a faculty member who is an elected senator subsequently elected by the
Senate to be its president (FS VII C.1.a). The President of the Senate
also chairs the Senate's Executive Committee. Until 2023, the University
Provost presided over the Senate, while the SEC was chaired by a faculty
member. As of this writing, the Senate President is Christopher (Chris)
Eliot
(\href{https://www.hofstra.edu/faculty-staff/faculty-profile.html?id=415}{contact
information}).

\section{What are senators-at-large?}\label{what-are-senators-at-large}

The Senate appoints to Senate committees others who are \emph{not}
voting members of the Senate. These appointed participants are called
``senators-at-large'' (styled thus in the Statute), and they can be
faculty, administration, chairpersons, or staff. Senators-at-large serve
for two year terms, with renewal possibilities. The purpose of having
this role is that the Senate have well-populated committees that are
capable of doing the main work of the Senate without making the Senate
itself, as a deliberative body, huge and unwieldy. Senators-at-large may
participate in the deliberations of the Senate, even though they only
vote in committee. (See FS VII B.3.) They play a crucial role in the
productivity of the Senate's committees.

\section{Who is on Senate committees, and in what
capacities?}\label{who-is-on-senate-committees-and-in-what-capacities}

The membership of each Senate committee is specified by its page in the
\href{https://www.hofstra.edu/sites/default/files/2022-04/senatebylaws.pdf}{Senate
By-laws}, and the current members of each committee are listed on the
Senate website, on the
\href{https://www.hofstra.edu/senate/committees-subcommittees.html}{committees
page}.

It is important to recognize, as a committee participant, that not
everyone who attends committee meetings attends in the same capacity or
has the same responsibilities there. Committees include elected senators
and appointed senators-at-large, and they are responsible for doing the
work of the committee. The senators and senators-at-large also elect one
of their members to be committee's chair. The committee sections of the
By-laws also specify an advisor for many of the committees who is
usually a member of the Provost's office. And many committees have
regular guests who also attend in an advisory capacity, people like the
Registrar or the Dean of Graduate Admissions.

\section{What do committee chairs
do?}\label{what-do-committee-chairs-do}

Each committee's chair is responsible for working with the Senate's
Senior Support Specialist to schedule the committee's meetings
---~ideally at a time when as many members as possible can attend. The
chair sets and circulates the agenda for each meeting, keep records on
each meeting in the form of minutes, and generally coordinates the work
of the committee.

Committee chairs take the lead in identifying issues and projects
relevant to their committee's areas of responsibility in the Senate
By-laws. They bring to committee meetings issues they identify
independently, issues that arise in discussion with others including
committee advisers, and issues referred to them by the Senate or the
Senate Executive Committee. Besides reviewing their committee's mandate
in the By-laws, committee chairs may find it useful to consult with
others around campus to identify possible issues.

Committee chairs also run meetings, ensuring there is a quorum (of at
least half of the elected and at-large members), coordinating
discussion, delegating tasks, and ultimately moving items to a vote when
they're ready.

Each chair of a standing committee of the Senate (UAAC, GAAC, FAC, P\&B,
and SAC) also automatically becomes a member of the Senate Executive
Committee (SEC), which coordinates the Senate's work and refers business
to committees as appropriate. As the SEC normally meets once a month
during the academic year, the standing committee chairs regularly report
on their committee's work to the SEC.

Chairs of special and subcommittees report regularly as in the By-laws.
as The chairs of subcommittees (ARC, CoL, CEP, CERT) normally report on
behalf of their committees to the parent committees at least once each
semester. The chairs of Senate special committees (COEC, SCAP, SCAC,
SCEHS) normally report in writing to the Senate Executive Committee once
a semester and may be asked by the Senate Executive Committee to report
on specific items directly to the Senate.

\section{What rules govern Senate
committees?}\label{what-rules-govern-senate-committees}

Senate committees are governed by the same rules as the Senate, but in
particular by the sections of the
\href{https://www.hofstra.edu/sites/default/files/2022-04/senatebylaws.pdf}{Senate
By-laws} establishing their committee. The By-laws set out the
membership and the mandate for each committee and sometimes other
guidance. Reading one's committee's brief By-laws carefully helps a lot
when the question arises, What are we doing here again?

Similarly, Senate committees are governed by \emph{Robert's Rules of
Order}. Some awkwardness in committee meetings and grumbling about rules
can be due to confusions about what \emph{Robert's Rules} involves. It
is widely overlooked that the \emph{Rules} acknowledges that in
committees ``some of the formality that is necessary in a large assembly
would hinder business'' (\emph{RRO} 49:21).

Specifically, in committee meetings, \emph{Robert's Rules} indicates:
``motions need not be seconded''; ``informal discussion of a subject is
permitted while no motion is pending''; ``when a proposal is perfectly
clear to all present, a vote can be taken without a motion's having been
introduced''; generally, ``a vote can be taken initially by a show of
hands''; and the chair ``may, without leaving the chair, speak in
informal discussions and in debate, and vote on all questions''
(\emph{RRO} 49:21). That is, the \emph{formal} guidelines for committees
are pretty casual. Confidently eschew excess formality.

Note that Hofstra Library has available ebooks of both
\href{https://hofstra.on.worldcat.org/oclc/1192561100}{\emph{Robert's
Rules of Order: Newly Revised}} and the very helpful, official
summary/overview version
\href{https://hofstra.on.worldcat.org/oclc/1192973323}{\emph{Robert's
Rules of Order: Newly Revised in Brief}}. (Incidentally, Hofstra
Professor of Mathematics Daniel Seabold is a co-author of both!) A
little familiarity with procedure can help one make things happen---not
only for committee chairs but also for participants.

\section{What is the Faculty Leadership
Group?}\label{what-is-the-faculty-leadership-group}

In 2013 a
\href{https://www.hofstra.edu/pdf/faculty/senate/senate_resolution_informational_meetings.pdf}{resolution}
was passed specifying that ``the {[}University{]} President shall meet
regularly (preferably at the beginning and the end of the fall and
spring semesters,) with the Speaker of the Faculty, the Chair of the
Senate Executive Committee, the Chair of the Chairs' Caucus and when
appropriate, other shared governance leadership, to provide
informational updates on developments concerning University strategies
and policies.''

Then in 2020 that small group of elected faculty leaders plus the AAUP
chapter president began meeting regularly with the Provost to address
the COVID crisis situation, and they have continued to meet. This
``FLG'' also meets with the Provost at least twice each semester to
discuss concerns, issues, ideas, and the like. Sometimes issues raised
in this forum get referred to Senate committees, just as others are
referred to appropriate administrators.

\section{How can elected senators communicate with their
constituents?}\label{how-can-elected-senators-communicate-with-their-constituents}

The Senate operates ``course'' sites on Hofstra's Canvas learning
management system for each of the faculty units that elect senators.
These sites aim to facilitate democratic engagement with the Senate and
raise awareness of Senate initiatives. Ideally, senators will share key
items of Senate business and news with constituents. Ideally
constituents will be reminded of who their senators are and think of
senators as potential recipients for items of concern and promising
ideas.

Senators are made ``teachers'' in of these Canvas courses, and all the
faculty in the unit are invited to the ``student'' role. This allows
senators to send (a) messages to their constituents by email or (b)
announcements on the course site that may also go out as emails,
depending on user preferences. Senators can open announcements to
constituent comments, and constituents can reply to messages or
announcements by email.

The membership of these sites should be refreshed every year.
Participants must opt in on Canvas to participate, and they must also
have ``receive announcements as emails'' turned on in their Canvas
settings to receive messages posted announcements.

For non-faculty student, staff, and administrative senators,
communication channels vary by group. (Senators should coordinate with
the Senate President and Senior Support Specialist if they need help.)

\end{document}
